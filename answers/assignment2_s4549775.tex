\documentclass[12pt, a4paper]{article}

\usepackage{amssymb}
\usepackage{enumerate}

\setlength\parskip{1em}
\setlength\parindent{0em}

\title{Assignment 2}

\author{Hendrik Werner s4549775}

\begin{document}
\maketitle

\section{} %1
\begin{enumerate}[a]
	\item %a
	Does not hold. Counterexample: $s_1 s_1 s_1 \dots$
	\item %b
	Holds. Example: $s_1 s_2 s_3 s_3 s_3 \dots$
	\item %c
	Holds. At every node there is a path to a circle where $\{a\}$ holds.
	\item %d
	Holds. We can enumerate the possibilities: $s_1 s_1$ holds, $s_1 s_2$ holds, $s_1 s_4$ holds.
	\item %e
	Does not hold. Counterexample: $s_1 s_1$.
	\item %f
	Holds. Every subexpression holds: $\exists \bigcirc a$ holds for $s_1 s_4$, $\exists \bigcirc b$ holds for $s_1 s_1$, $\exists \bigcirc c$ holds for $s_1 s_2$.
	\item %g
	Holds. For every node there exists a path to a node for which $c$ holds.
	\item %h
	Holds. For every node there is at least one edge to a node where $a$ holds.
	\item %i
	Holds. $s_1 s_2 \dots$ trivially holds, and $s_1 s_4 \dots$ holds because $b$ holds for $s_1$, and $a$ holds for every node reachable from $s_4$.
	\item %j
\end{enumerate}

\section{} %2
\begin{enumerate}[a]
	\item %a
	$\forall \bigcirc \forall \square \psi = \forall \square \forall \bigcirc \psi$

	$\forall \bigcirc \forall \square \psi$ means "For every next step, $\psi$ holds for all paths." $\forall \square \forall \bigcirc \psi$ means "For all paths, $\psi$ always holds for the next step.".

	Both are equivalent to $\forall \square \psi$.
	\item %b
	\item %c
	\item %d
\end{enumerate}

\section{} %3
\begin{enumerate}[a]
	\item %a
	$\forall \square \lnot (run_1 \land run_2)$
	\item %b
	$\exists \diamond (run_1 \land run_2)$
	\item %c
	$\forall \square \forall \diamond run_1$
	\item %d
	$\forall \square \exists \diamond run_1$
	\item %e
	$\forall \square ((\lnot run_1 \cup run_2) \land (\lnot run_2 \cup run_1))$
	\item %f
	This cannot be expressed in CTL because the "if \dots, then \dots" cannot be expressed in CTL. You would need an implication of the form $"\dots \rightarrow \square \dots"$ or a weak until. Both of them are not available in CTL.

	In LTL it can be axpressed as $run_1 W run_2$.
	\item %g
	This cannot be expressed in CTL because the "if \dots, then \dots" cannot be expressed in CTL. You would need an implication of the form $"\dots \rightarrow \diamond \dots"$ .

	It can also not be expressed in LTL because there is no "possibly", due to LTL dealing with single branches.
	\item %h
	$(\forall \square \exists \diamond stop_1) \land (\forall \square \exists \diamond stop_2)$
	\item %i
	This cannot be expressed in CTL because the "if \dots, then \dots" cannot be expressed in CTL. You would need an implication of the form $"\dots \rightarrow \square \dots"$.
	\item %j
	$\forall \square (run_1 \rightarrow \exists (\lnot run_1 \cup run_2))$
\end{enumerate}

\end{document}
